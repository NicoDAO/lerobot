% !TEX encoding = UTF-8 Unicode
\newpage

\chapter{Linux}



\section{Liens utiles}

https://blog.mbedded.ninja/programming/embedded-linux/zynq/building-linux-for-the-zynq-zc702-eval-kit-using-yocto/#&gid=1&pid=1

https://www.yoctoproject.org/docs/2.4/dev-manual/dev-manual.html

\section{Génération du bootloader FSBL}



/home/nicolas/vivado/vivado/FPGA_xilinx/vivado2019_2/workspace/design_1_wrapper_corona2/export/design_1_wrapper_corona2/sw/design_1_wrapper_corona2/boot/fsbl.elf

/home/nicolas/vivado/vivado/FPGA_xilinx/vivado2019_2/workspace/dsfyoyo/Debug/dsfyoyo.elf

/home/nicolas/vivado/vivado/FPGA_xilinx/vivado2019_2/workspace/design_1_wrapper_corona2/bitstream/design_1_wrapper_corona2.bit


\subsubsection{Génération boot.bin avec Xilinx Vivado}

https://www.xilinx.com/support/documentation/sw_manuals/xilinx2019_2/ug1283-bootgen-user-guide.pdf

\\
\fbox{\begin{minipage}{0.9\textwidth}
nicolas@debianicolas:~/yocto/lerobot/generation_boot$ more genere.bif

//arch = zynq; split = false; format = BIN

the_ROM_image:

{

[bootloader]u-boot-zybo-zynq7.elf

design_1_wrapper.bit

}

nicolas@debianicolas:~/yocto/lerobot/generation_boot$ /home/nicolas/Xilinx/Vivado/Vivado/2020.1/bin/bootgen -image genere.bif -o boute.bin -w

\end{minipage}}

\subsubsection{Résultat génération boot.bin}

On obtient alors le résultat : 
\\
\fbox{\begin{minipage}{0.9\textwidth}
'
****** Xilinx Bootgen v2020.1

**** Build date : May 27 2020-20:33:36

** Copyright 1986-2020 Xilinx, Inc. All Rights Reserved.

[INFO] : Bootimage generated successfully
Le .bit est le fichier généré par vivado2019



le.elf est le fichier de boot généré par vitis, mais ici récupéré chez Xilinx (pb d’horloge)

\end{minipage}}



\subsection{creer bootgen avec le script genere.sh}


\fbox{\begin{minipage}{0.9\textwidth}
#!/bin/sh                                                                       
/home/nicolas/Xilinx/Vivado/Vivado/2020.1/bin/bootgen -arch zynq -image genere.\
bif -o boot.bin -w

\end{minipage}}


\section{flash l’image sur la carte SD}

La distribution Linux doit être flashée sur une carte SD, afin que la carte du Robot puisse booter dessus.
La carte doit être configurée en 2 partitions, celle contenant le boot, et celle contenant le root.
La carte SD doit être bootable.

\subsection{creer bootgen avec le script genere.sh}
Afin de copier l'image Linux générée pat Yocto, on commence par formatter la carte SD avec la commande : 

\textit{sudo dd if=/dev/zero of=/dev/mmcblk0 of=/dev/
}
\subsection{configuration des partitions}

La carte SD a besoin de 2 partitions, créées avec la commande : 

\textit{sudo fdisk /dev/mmcblk0
}


Tapez les commandes suivantes : 


\fbox{\begin{minipage}{0.9\textwidth}

Command (m for help): n

Partition type: p primary (0 primary, 0 extended, 4 free) e extended Select (default p): p

Partition number (1-4, default 1): 1

First sector (2048-15759359, default 2048): Using default value 2048

Last sector, +sectors or +size{K,M,G} (2048-15759359, default 15759359): +200M




Command (m for help): n

Partition type: p primary (1 primary, 0 extended, 3 free) e extended Select (default p): p

Partition number (1-4, default 2): 2

First sector (411648-15759359, default 411648): Using default value 411648

Last sector, +sectors or +size{K,M,G} (411648-15759359,

default 15759359): Using default value 15759359





Command (m for help): a

Partition number (1-4): 1



Command (m for help): t

Partition number (1-4): 1

Hex code (type L to list codes): c

Changed system type of partition 1 to c (W95 FAT32 (LBA))



Command (m for help): t

Partition number (1-4): 2

Hex code (type L to list codes): 83



Commande (m pour l'aide) : p

Disque /dev/mmcblk0 : 7,4 GiB, 7948206080 octets, 15523840 secteurs

Unités : secteur de 1 × 512 = 512 octets

Taille de secteur (logique / physique) : 512 octets / 512 octets

taille d'E/S (minimale / optimale) : 512 octets / 512 octets

Type d'étiquette de disque : dos

Identifiant de disque : 0xca4a050b



Périphérique Amorçage Début Fin Secteurs Taille Id Type

/dev/mmcblk0p1 * 2048 411647 409600 200M c W95 FAT32 (LBA)

/dev/mmcblk0p2 411648 15523839 15112192 7,2G 83 Linux





Command (m for help): w

\end{minipage}}

\subsection{création des systèmes de fichier}

Il faut créer les systèmes de fichiers avec les commandes suivantes : 


\textit{sudo mkfs.vfat -F 32 -n boot /dev/mmcblk0p1}

\textit{sudo mkfs.ext4 -L root /dev/mmcblk0p2}



\subsection{Copie des images disques sur  la carte SD}


Il faut générer la distribution Linux, et ensuite la claquer sur la carte SD

Si ce n'est pas fait, cloner le projet dans un répertoire
\\

\textit{git clone -b dunfell git://git.yoctoproject.org/poky.git}
\\
Aller dans le répertoire poky

\textit{source oe-init-build-env}

Dans le fichier \textit{local.conf} il faut rajouter : 
\begin{itemize}
\item ssh-server-openssh, afin de se connecter en ssh sur la cible
\item tools-sdk, afin de compiler le logiciel sur la cible
\item tools-debug, afin d'avoir les outils de debuf (gdb) sur la cible

\end{itemize}

\fbox{\begin{minipage}{0.9\textwidth}
# Additional image features\\
#\\
EXTRA_IMAGE_FEATURES ?= "debug-tweaks"\\
EXTRA_IMAGE_FEATURES += "ssh-server-openssh"\\
EXTRA_IMAGE_FEATURES += "tools-sdk"\\
EXTRA_IMAGE_FEATURES += "tools-debug"\\
\textit{source oe-init-build-env} pour charger l'environnement de compilation

\end{minipage}}

\subsection{mettre le nouveau file system}


\textit{sudo tar x -C /media/nicolas/root/ -f core-image-minimal-zybo-zynq7.tar.gz}

\textit{sudo tar x -C /mnt/mmcblk0p2/ -f core-image-minimal-zybo-zynq7.tar.gz}



copier le contenu du répertoire dans la partition boot

cp * /media/nicolas/boot/

sudo mount /dev/mmcblk0p2 /mnt/mmcblk0p2





**************compiler

~/poky/meta-xilinx$ cd ..

~/poky$

bitbake-layers add-layer "$HOME/poky/meta-xilinx"

rajouter MACHINE ?= "zybo-zynq7" dans local.conf

bitbake-layers add-layer "$HOME/yocto/poky/meta-xilinx"





**************generation du boot.bin avec le bitstream

//arch = zynq; split = false; format = BIN

the_ROM_image:

{

[bootloader]/home/nicolas/vivado/vivado/FPGA_xilinx/vivado2019_2/workspace/dsfyoyo/Debug/dsfyoyo.elf

/home/nicolas/vivado/vivado/FPGA_xilinx/vivado2019_2/workspace/dsfyoyo/_ide/bitstream/design_1_wrapper_corona2.bit

/home/nicolas/yocto/poky/build/tmp/deploy/images/zybo-zynq7/u-boot.elf

/home/nicolas/yocto/poky/build/tmp/deploy/images/zybo-zynq7/u-boot-zybo-zynq7.elf

}





cp *.bin /mnt/mmcblk0p1

cp *.img /mnt/mmcblk0p1

cp *.dtb /mnt/mmcblk0p1

cp *.elf /mnt/mmcblk0p1

cp *.u-boot /mnt/mmcblk0p1



bitbake -c menuconfig virtual/kernel





http://openpowerlink.sourceforge.net/web/openPOWERLINK/Getting%20Started/Zynq%20/%20Petalinux.html



