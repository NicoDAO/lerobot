% !TEX encoding = UTF-8 Unicode

\chapter{Intro}

\section{Présentation Robot}
\lettrine[lines=1]{L}e robot peut fonctionner soit en mode autonome, soit en mode semi-autonoeme, c'est à dire qu'il est piloté par une appli Android




\section{Architecture du projet}

Les axes de développement du projet sont
\begin{itemize}
\item  Le linux embarqué
J'utilise Yocto Linux pour faire tourner le logiciel du robot. Le Linux est généré et claqué sur une carte SD.
Son rôle est de lancer le logiciel du robot, communiquer de manière sécurisée sur le réseau avec un protocole sécurisé comme SSH, gérer les mises à jours.

\item Logiciel du robot est constitué de programmes qui s'exécutent sur la cible et qui permettent de faire tourner le robot.
Ces Logiciels permettent de piloter les moteurs, traiter les informations des différents capteurs, gérer le mode autonome du robot, et communiquer avec la télécommande.
\item L'électronique et la gestion des couches basses se font dans la partie FPGA du composant.
Tous ce qui est gestion des périphériques se fait matériellement, comme la gestion des signaux électriques, les protocoles d'échange avec les différents capteurs et actionneurs.
Le logiciel communique avec le FPGA avec un protocole allégé.

\item L'intégration mécanique est pour le moment hyper rudimentaire

\end{itemize}




L'arborescence contiendra donc les répertoires suivants : 




\begin{itemize}


\item 


\subsection{Vivado}
  : le projet vivado pour la partie FPGA

\end{itemize}



 