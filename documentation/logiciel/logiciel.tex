% !TEX encoding = UTF-8 Unicode
\newpage

\chapter{Logiciel}
\section{Logiciel}
\lettrine[lines=1]{L}e robot peut fonctionner soit en mode autonome, soit en mode semi-autonome, c'est à dire qu'il est piloté par une appli Android


\subsection{Compilation du logiciel}



le logiciel s ‘exécute sur la cible. Il est téléchargé par scp, puis recompilé par la commande

\textit{sh compile\_simone.sh}



mise au point avec gdb

\textit{gdb sorie}

\textit{(gdb) set args simuPC = lance en mode simulation pc}



\subsection{Architecture  du logiciel}


\subsection{Execution du logiciel}





mode simulation





permet de simuler le fonctionnement des capteurs et ainsi tetster différents scénarios. Par exemple tester un scénario avec le capteur de distance.

Pour cela taper :




 